% Template for Cogsci submission with R Markdown

% Stuff changed from original Markdown PLOS Template
\documentclass[10pt, letterpaper]{article}

\usepackage{cogsci}
\usepackage{pslatex}
\usepackage{float}
\usepackage{caption}

% amsmath package, useful for mathematical formulas
\usepackage{amsmath}

% amssymb package, useful for mathematical symbols
\usepackage{amssymb}

% hyperref package, useful for hyperlinks
\usepackage{hyperref}

% graphicx package, useful for including eps and pdf graphics
% include graphics with the command \includegraphics
\usepackage{graphicx}

% Sweave(-like)
\usepackage{fancyvrb}
\DefineVerbatimEnvironment{Sinput}{Verbatim}{fontshape=sl}
\DefineVerbatimEnvironment{Soutput}{Verbatim}{}
\DefineVerbatimEnvironment{Scode}{Verbatim}{fontshape=sl}
\newenvironment{Schunk}{}{}
\DefineVerbatimEnvironment{Code}{Verbatim}{}
\DefineVerbatimEnvironment{CodeInput}{Verbatim}{fontshape=sl}
\DefineVerbatimEnvironment{CodeOutput}{Verbatim}{}
\newenvironment{CodeChunk}{}{}

% cite package, to clean up citations in the main text. Do not remove.
\usepackage{apacite}

% KM added 1/4/18 to allow control of blind submission


\usepackage{color}

% Use doublespacing - comment out for single spacing
%\usepackage{setspace}
%\doublespacing


% % Text layout
% \topmargin 0.0cm
% \oddsidemargin 0.5cm
% \evensidemargin 0.5cm
% \textwidth 16cm
% \textheight 21cm

\title{}





\begin{document}

\maketitle

\begin{abstract}


\textbf{Keywords:}

\end{abstract}

\hypertarget{introduction}{%
\section{Introduction}\label{introduction}}

\hypertarget{methods}{%
\section{Methods}\label{methods}}

\hypertarget{corpus}{%
\subsection{Corpus}\label{corpus}}

\hypertarget{manual-annotation}{%
\subsection{Manual annotation}\label{manual-annotation}}

\hypertarget{reliability}{%
\subsection{Reliability}\label{reliability}}

\hypertarget{results}{%
\section{Results}\label{results}}

\hypertarget{overall-characteristics-of-object-handling}{%
\subsection{Overall characteristics of object
handling}\label{overall-characteristics-of-object-handling}}

On average, children handled 4.54 objects per hour (median = 4.39,
\emph{SD} = 2.63, range = 1--10). Across the whole waking day, children
handled an average of 21.16 unique objects (median = 20, \emph{SD} =
15.2, range = 1--59), with no significant differences across sites
(\emph{M}\textsubscript{Rossel} = 18.93, \emph{M}\textsubscript{Tseltal}
= 23.24, \emph{W} = 350, \emph{p} = 0.501).

The frequency of object categories was similarly divided across
sites\ldots{}

\hypertarget{time-of-day-effects}{%
\subsection{Time of day effects}\label{time-of-day-effects}}

\hypertarget{age-effects}{%
\subsection{Age effects}\label{age-effects}}

\hypertarget{discussion}{%
\section{Discussion}\label{discussion}}

\hypertarget{references}{%
\section{References}\label{references}}

\setlength{\parindent}{-0.1in} 
\setlength{\leftskip}{0.125in}

\noindent

\bibliographystyle{apacite}


\end{document}
