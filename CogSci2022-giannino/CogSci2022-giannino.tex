% Template for Cogsci submission with R Markdown

% Stuff changed from original Markdown PLOS Template
\documentclass[10pt, letterpaper]{article}

\usepackage{cogsci}
\usepackage{pslatex}
\usepackage{float}
\usepackage{caption}

% amsmath package, useful for mathematical formulas
\usepackage{amsmath}

% amssymb package, useful for mathematical symbols
\usepackage{amssymb}

% hyperref package, useful for hyperlinks
\usepackage{hyperref}

% graphicx package, useful for including eps and pdf graphics
% include graphics with the command \includegraphics
\usepackage{graphicx}

% Sweave(-like)
\usepackage{fancyvrb}
\DefineVerbatimEnvironment{Sinput}{Verbatim}{fontshape=sl}
\DefineVerbatimEnvironment{Soutput}{Verbatim}{}
\DefineVerbatimEnvironment{Scode}{Verbatim}{fontshape=sl}
\newenvironment{Schunk}{}{}
\DefineVerbatimEnvironment{Code}{Verbatim}{}
\DefineVerbatimEnvironment{CodeInput}{Verbatim}{fontshape=sl}
\DefineVerbatimEnvironment{CodeOutput}{Verbatim}{}
\newenvironment{CodeChunk}{}{}

% cite package, to clean up citations in the main text. Do not remove.
\usepackage{apacite}

% KM added 1/4/18 to allow control of blind submission


\usepackage{color}

% Use doublespacing - comment out for single spacing
%\usepackage{setspace}
%\doublespacing


% % Text layout
% \topmargin 0.0cm
% \oddsidemargin 0.5cm
% \evensidemargin 0.5cm
% \textwidth 16cm
% \textheight 21cm

\title{}





\begin{document}

\maketitle

\begin{abstract}


\textbf{Keywords:}

\end{abstract}

\hypertarget{introduction}{%
\section{Introduction}\label{introduction}}

\hypertarget{methods}{%
\section{Methods}\label{methods}}

\hypertarget{corpus}{%
\subsection{Corpus}\label{corpus}}

\hypertarget{manual-annotation}{%
\subsection{Manual annotation}\label{manual-annotation}}

\hypertarget{reliability}{%
\subsection{Reliability}\label{reliability}}

\hypertarget{results}{%
\section{Results}\label{results}}

\hypertarget{overall-characteristics-of-object-handling}{%
\subsection{Overall characteristics of object
handling}\label{overall-characteristics-of-object-handling}}

Across the entire waking day, children handled an average of 21.16
unique objects (median = 20, \emph{SD} = 15.2, range = 1--59), with no
significant differences across sites
(\emph{M}\textsubscript{\emph{Rossel}} = 18.93,
\emph{M}\textsubscript{\emph{Tseltal}} = 23.24, \emph{W} = 350, \emph{p}
= 0.501). During any given hour, children handled 5.26 objects, on
average (median = 4.5, \emph{SD} = 3.92, range = 1--18).

The frequency of object categories was similarly divided across sites
(Figure 1a). Children primarily handled miscellaneous synthetic objects
(e.g., rope, guitar, shirt, etc.; \emph{M}\textsubscript{\emph{Rossel}}
= 32.01\% of handling, \emph{M}\textsubscript{\emph{Tseltal}} = 37.5\%)
and food (\emph{M}\textsubscript{\emph{Rossel}} = 28.58\%,
\emph{M}\textsubscript{\emph{Tseltal}} = 36.21\%). For 45 of 56
children, their top category was either synthetic objects or food.
Two-tailed Wilcoxon tests revealed that the only significant difference
between sites was seen in children's handling of large or immovable
objects (e.g., veranda, ladder, railing, etc.), where Rossel children
handled these objects more frequently than Tseltal children
(\emph{M}\textsubscript{\emph{Rossel}} = 7.73\%,
\emph{M}\textsubscript{\emph{Tseltal}} = 3.31\%, adjusted \emph{p} =
0.038, \emph{p}s for all other categories \textgreater{} 0.05).

\hypertarget{time-of-day-effects}{%
\subsection{Time of day effects}\label{time-of-day-effects}}

\hypertarget{age-effects}{%
\subsection{Age effects}\label{age-effects}}

\hypertarget{discussion}{%
\section{Discussion}\label{discussion}}

\hypertarget{references}{%
\section{References}\label{references}}

\setlength{\parindent}{-0.1in} 
\setlength{\leftskip}{0.125in}

\noindent

\bibliographystyle{apacite}


\end{document}
